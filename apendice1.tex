\documentclass[11pt, twoside]{book}
\usepackage[utf8]{inputenc}
\usepackage[OT1]{fontenc}
\usepackage{imunam-tesis}
\usepackage[subpreambles=false]{standalone}
\usepackage[spanish, mexico]{babel}
\usepackage{mathpazo} %Fuente principal
\usepackage{eulervm} %Fuente en el entorno matemático
\usepackage{mathtools} % Para las flechas de inclusión 

\usepackage{float}

% commando necesario para generar los índices
\makeindex

% Paquete para hace gráficas
\usepackage{pgfplots}
\pgfplotsset{compat=1.11} 

% Paquete para incluir dos subfiguras en una misma figura
\usepackage{subcaption}

% Configura tikz para los diagramas
\usepackage{tikz}
\usetikzlibrary{cd}
\usetikzlibrary{arrows}
\usetikzlibrary{decorations.markings}
\tikzset{
commutative diagrams/.cd,
arrow style=tikz,
diagrams={>=stealth}}

% Compila la bibliografia con biblatex
\usepackage[
  style=ieee,
  backend=biber,
  sorting=ynt
]{biblatex}

\addbibresource{bibliografia.bib}

\graphicspath{{imagenes/}{../imagenes/}}

% Usa romanos en las listas enumeradas
\renewcommand{\labelenumi}{\roman{enumi}}


\begin{document}
\chapter{Lema de escisión para una sucesión exacta corta}\label{apendice1}

El apéndice reinicia los contadores. Por ejemplo, nótese cómo es enumerado el
siguiente teorema

\begin{teorema}[Lema de separación] \label{separacion1}
  Sea
  \[1 \rightarrow N \xrightarrow{\phi} G\xrightarrow{\sigma} H \rightarrow 1\]
  una sucesión exacta corta. Las siguientes son equivalentes:
  \begin{enumerate}
  \item Existe un homomorfismo \(s:H \rightarrow G\) tal que \(\sigma(s(h)) = h
    \), \(\forall h \in H\).
  \item Existe una acción \(\phi:N \times H \rightarrow N\) tal que el
    producto semidirecto \(N \rtimes_{\phi} H\) es isomorfo a \(G\).
  \end{enumerate}
\end{teorema}

\end{document}
