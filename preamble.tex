\usepackage[subpreambles=false]{standalone}
\usepackage[spanish, mexico]{babel}
\usepackage{mathpazo} %Fuente principal
\usepackage{eulervm} %Fuente en el entorno matemático
\usepackage{mathtools} % Para las flechas de inclusión 

\usepackage{float}

% commando necesario para generar los índices
\makeindex

% Paquete para hace gráficas
\usepackage{pgfplots}
\pgfplotsset{compat=1.11} 

% Paquete para incluir dos subfiguras en una misma figura
\usepackage{subcaption}

% Configura tikz para los diagramas
\usepackage{tikz}
\usetikzlibrary{cd}
\usetikzlibrary{arrows}
\usetikzlibrary{decorations.markings}
\tikzset{
commutative diagrams/.cd,
arrow style=tikz,
diagrams={>=stealth}}

% Compila la bibliografia con biblatex
\usepackage[
  style=ieee,
  backend=biber,
  sorting=ynt
]{biblatex}

\addbibresource{bibliografia.bib}

\graphicspath{{imagenes/}{../imagenes/}}

% Usa romanos en las listas enumeradas
\renewcommand{\labelenumi}{\roman{enumi}}
